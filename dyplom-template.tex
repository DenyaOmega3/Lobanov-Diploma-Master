\documentclass{thesis}

\renewcommand{\baselinestretch}{1.20}
\DeclareMathOperator{\tr}{tr}


\begin{document}
\allowdisplaybreaks

\large

\setcounter{page}{1}
\thispagestyle{empty}
\centerline{Національна академія наук України}
\centerline{Міністерство освіти і науки України}
\centerline{Державна наукова установа «Київський академічний університет»}

\vspace{10mm}

\begin{flushright}
\begin{minipage}{100mm}
\begin{center}\large {\bf <<Допущено до захисту>>}\\
Завідувач кафедри математики,\\
доктор фіз.-мат. наук\\
{\bf Вячеслав БОЙКО}\\
<<\underline{\hspace{8mm}}>> травня 2025 р.
\end{center}
\end{minipage}
\end{flushright}

\vspace{10mm}

\centerline{\Large \bf Лобанов Денис Артемович}

\begin{center}
{\bf КВАЛІФІКАЦІЙНА РОБОТА}\\
на здобуття освітнього ступеня <<магістр>>\\
Спеціальність 111 <<Математика>>\\[4mm]
{\Large \bf Тема: <<Опис кубічних кілець зі скінченним числом незвідних зображень>>}
\end{center}


\vspace{5mm}

\noindent
{Засвідчую, що кваліфікаційна робота містить результати власних досліджень. Використання ідей, результатів і~текстів інших авторів мають посилання на відповідне джерело.
\underline{\hspace{18mm}} Д.А.~ЛОБАНОВ\par}

\vspace{5mm}

\begin{flushright}
\begin{minipage}{90mm}
\large {\bf Науковий керівник}\\
доктор фіз.-мат. наук, професор\\
{\bf Дрозд Юрій Анатолійович}\\
\underline{\hspace{48mm}}

\end{minipage}
\end{flushright}


\vfill

\centerline{\bf Київ --- 2025}

\newpage

\begin{center}
\Large \bf Анотація
\end{center}

\noindent
\textbf{Лобанов Д.А.}, \textbf{Опис кубічних кілець зі скінченним числом незвідних зображень}, Кваліфікаційна робота на здобуття освітнього ступеня <<магістр>> за спеціальністю 111 Математика, Київський академічний університет, Київ, 2025, ??~с., ??~джерел.

\bigskip


???Текст анотації???

\bigskip

\noindent
{\bf MSC:} ???????

\bigskip

\noindent
{\bf Ключові слова:} ???????
\bigskip

\newpage

\begin{center}
\Large \bf Abstract
\end{center}

\noindent
\textbf{LastName N.S.}, \textbf{?????????}, Master Thesis, speciality 111 Mathematics.~--
Kyiv Academic University, Kyiv, 2025, ??~pages, ??~references.

\bigskip
??????

\bigskip

\noindent
{\bf MCS:} ??????? %for 111, see https://mathscinet.ams.org/mathscinet/msc/msc2020.html

%\noindent
%{\bf ACM:} ??????? %for 122, see https://cran.r-project.org/web/classifications/ACM.html

\bigskip

\noindent
{\bf Key words:} ???????


\newpage


\tableofcontents

\newpage

\phantomsection
\section*{Перелік умовних позначень}
\addcontentsline{toc}{chapter}{Перелік умовних позначень}

\bigskip


\begin{tabular}{ll}
$\mathbb{Z}$ & кільце цілих чисел\\
$\Lambda$ & кубічне кільце\\
$f$ & кубічна форма (індекс-форма) кільця\\
$\Delta$ & дискримінант кубічної форми\\
$G$ & група Лі\\
$\mathfrak g$ & алгебра Лі\\
$Q$ & оператор однопараметричної групи Лі\\[1mm]
$\mathbb R^n$ & $n$-вимірний евклідів простір\\[1mm]
$X\simeq {\mathbb R}^n$ & простір незалежних змінних~$x=(x^1,x^2,\ldots, x^n)$ \\[1mm]
$U\simeq {\mathbb R}^m$ & простір залежних змінних $u=\big(u^1, u^2, \ldots, u^m\big)$\\[1mm]
$u^\alpha_i=\dfrac{\partial u^\alpha}{\partial x^i}$ & частинна похідна від залежної змінної $u^\alpha$\\[1mm]
& за незалежною змінною $x^i$\\[1mm]
$D_{i}$ & оператор повної похідної за змінною $x^{i}$\\[1mm]
$\underset{r}{Q}$ & $r$-те продовження оператора $Q$\\[1mm]
$I$ & набір інваріантів нульового порядку\\[1mm]
$I_{(r)}$ & набір інваріантів $r$-го порядку, $r\geq 1$

\end{tabular}


\newpage

\phantomsection
\chapter*{Вступ}\label{Introduction}
\addcontentsline{toc}{chapter}{Вступ}

??????????
\cite{boyko-thesis,boyko2021,PopovychBoykoNesterenkoLutfullin2003}

\newpage

\chapter{Максимальне кубічне кільце}\label{chaper1}

\section{Попередні відомості про кубічні кільця}\label{section1.1}
Необхідно пригадати важливі означення та результати.
\begin{definition}
\textbf{Кубічним кільцем} $\Lambda$ називають вільний $\mathbb{Z}$-модулем рангу $3$.
\end{definition}

\begin{remark}
Розглянемо базис $\{\omega_0,\omega_1,\omega_2\}$ кубічного кільця $\Lambda$. Тоді ми можемо завжди обрати базис так, щоб перший елемент був $1$. Тому надалі обирамємо базис $\{1,\omega_1,\omega_2\}$.
\end{remark}

\begin{definition}
Базис $\{1,\omega_1,\omega_2\}$ кубічного кільця $\Lambda$ називається \textbf{нормальним}, якщо
\begin{center}
$\omega_1 \cdot \omega_2 \in \mathbb{Z}$
\end{center}
\end{definition}

\begin{remark}
Зафіксуємо базис $\{1,\omega_1,\omega_2\}$ кільця $\Lambda$. Тоді $\omega_1 \cdot \omega_2 = k \omega_1 + l \omega_2 + m$ при $k,l,m \in \mathbb{Z}$. Замінивши $\omega_1$ на $\omega_1-l$ та $\omega_2$ на $\omega_2 - k$, отримуємо нормальний базис $\{1,\omega_1 - l, \omega_2 - k\}$. Отже, надалі всі базиси в кубічних кільцях -- нормальні.
\end{remark}

Запишемо таблицю множення для базиса $\{1,\omega_1,\omega_2\}$:
\begin{align*}
\omega_1^2 & = b \omega_1 + a \omega_2 + r \\
\omega_1 \omega_2 & = s \\
\omega_2^2 & = d \omega_1 + c \omega_2 + t
\end{align*}
Прирівнявши $(\omega_1^2) \cdot \omega_2 = \omega_1 \cdot (\omega_1 \cdot \omega_2)$ та $\omega_1 \cdot (\omega_2^2) = (\omega_1 \cdot \omega_2) \cdot \omega$, отримуємо співвідношення $r = -ac,\ s = ad,\ t = -bd$. Отже, таблиця множення задається чотирма числами $a,b,c,z \in \mathbb{Z}$:
\begin{align*}
\omega_1^2 & = b \omega_1 + a \omega_2 - ac \\
\omega_1 \omega_2 & = ad \\
\omega_2^2 & = d \omega_1 + c \omega_2 + -bd
\end{align*}

\section{Критерій максимальності кільця}
\begin{theorem}
Кільце $\Lambda$ -- не максимальне $\iff$ існує такий цілий елемент $\theta$, що $\theta \notin \Lambda$. 
\end{theorem}

Варто відповісти на питання, що означає, що $\theta$ -- цілий елемент. Розглянемо оператор множення на $\theta$ (тобто $\lambda \mapsto \theta \lambda$). Запишемо матрицю $[\theta]$ оператора множення. Розглянемо характеристичний многочлен $\chi_{[\theta]}(t)$ даної матриці. Тоді $\theta$ називають \textbf{цілим}, якщо коефіцієнти характеристичного многочлен $\chi$ -- цілі числа.

Нагадаємо, що для випадку матриці $3 \times 3$ характеристичний многочлен має вигляд
\begin{align*}
\chi_{[\theta]}(t) = \det [\theta] - M_{2 \times 2}[\theta] \cdot t + \tr [\theta] t^2 - t^3,
\end{align*}
де
\begin{align*}
M_{2 \times 2}[\theta] = M_{12}^{12}[\theta] + M_{13}^{13}[\theta] + M_{23}^{23}[\theta]
\end{align*}

Розглянемо випадок, коли $\theta = \dfrac{\alpha + \beta \omega_1}{p}$. Знайдемо матрицю $[\theta]$ оператора множення в базисі $\{1,\omega_1,\omega_2\}$.
\begin{align*}
\theta \cdot 1 = \dfrac{\alpha + \beta \omega_1 p}{p} = \dfrac{\alpha}{p} \cdot 1 + \dfrac{\beta}{p} \omega_1 + 0 \cdot \omega_2
\end{align*}
\begin{align*}
\theta \cdot \omega_1 & = \dfrac{\alpha + \beta \omega_1}{p} \cdot \omega_1 = \dfrac{\alpha \omega_1}{p} + \dfrac{\beta (-ac + b \omega_1 + a \omega_2)}{p} = \\
& = \dfrac{-\beta a c}{p} \cdot 1 + \dfrac{\alpha + \beta b}{p} \cdot \omega_1 + \dfrac{\beta a}{p} \cdot \omega_2
\end{align*}
\begin{align*}
\theta \cdot \omega_2 = \dfrac{\alpha + \beta \omega_1}{p} \cdot \omega_2 = \dfrac{\alpha \omega_2 + \beta ad}{p} = \dfrac{\beta a d}{p} \cdot 1 + 0 \cdot \omega_1 + \dfrac{\alpha}{p} \cdot \omega_2
\end{align*}

Зібравши всі коефіцієнти, отримуємо такий вигляд матриці:
\begin{align*}
[\theta] = \dfrac{1}{p} \begin{pmatrix}
\alpha & -\beta a c & \beta a d \\
\beta & \alpha + \beta b & 0 \\
0 & \beta a & \alpha
\end{pmatrix}
\end{align*}

Оскільки наша мета -- це $\theta$ бути цілим, треба вимагати, щоб характеристичний многочлен мав цілі коефіцієнти. Зокрема необхідно розглянути $\tr [\theta],\ \det [\theta]$, а також $M_{2 \times 2}[\theta]$.

\begin{align*}
\tr [\theta] = \dfrac{1}{p} (3 \alpha + \beta b) \\
\det [\theta] = \dfrac{1}{p^3} \left( \alpha^3 + \alpha^2 \beta b + \beta^3 a^2 d + \alpha \beta^2 ac \right)
\end{align*}
Детальніше розпишемо знаходження $M_{2 \times 2}[\theta]$. Маємо
\begin{align*}
M_{2 \times 2}[\theta] & = M_{12}^{12}[\theta] + M_{13}^{13}[\theta] + M_{23}^{23}[\theta] = \\
& = \dfrac{1}{p^2} \det \begin{pmatrix}
\alpha & -\beta \alpha c \\
\beta & \alpha + \beta b
\end{pmatrix} + \dfrac{1}{p^2} \det \begin{pmatrix}
\alpha & \beta a \alpha \\
0 & \alpha
\end{pmatrix} + \dfrac{1}{p^2} \det \begin{pmatrix}
\alpha + \beta b & 0 \\
\beta a & \alpha
\end{pmatrix} = \\
& = \dfrac{1}{p^2} \left( \alpha^2 + \alpha \beta b + \beta^2 \alpha c + \alpha^2 + \alpha^2 + \alpha \beta b \right) = \\ 
& = \dfrac{1}{p^2} \left( 3 \alpha^2 + \beta^2 a c + 2 \alpha \beta b\right)
\end{align*}

Наша вимога -- це $\tr [\theta],\ \det [\theta],\ M_{2 \times 2}[\theta] \in \mathbb{Z}$. Запишемо умову в термінах конгруентних рівнянь:
\begin{equation}
    \begin{cases}
      \alpha^3 + \alpha^2 \beta b + \beta^3 a^2 d + \alpha \beta^2 a c & \equiv 0 \pmod {p^3} \\
    3\alpha^2 + \beta^2 ac + 2 \alpha \beta b & \equiv 0 \pmod {p^2} \\
      3 \alpha + \beta b & \equiv 0 \pmod p 
    \end{cases}\,.
\end{equation}
Припускаючи, що $p$ -- просте число та $\alpha \neq 0$, ми можемо підібрати елемент $\alpha^{-1}$ -- оборотний елемент класу лишків $\mathbb{Z}_p$ відносно множення. Третє, друге, перше рівняння домножуємо на відповідно $\alpha^{-1},\ \alpha^{-2},\ \alpha^{-3}$ -- отримуємо наступне:
\begin{equation*}
    \begin{cases}
      1 + (\beta \alpha^{-1}) b + (\beta \alpha^{-1})^{3} a^2 d + (\beta \alpha^{-1})^2 a c & \equiv 0 \pmod {p^3} \\
    3 + ((\beta \alpha^{-1})^2 ac + 2 (\beta \alpha^{-1}) b & \equiv 0 \pmod {p^2} \\
      3 + (\beta \alpha^{-1})b & \equiv 0 \pmod p 
    \end{cases}\,.
\end{equation*}
Проведемо заміну:
\begin{align}
t = \beta \alpha^{-1}
\end{align}
Задача звелась до розв'язання системи конгруентних рівнянь відносно $t$, якщо $p$ -- просте число: 
\begin{equation}
\label{system_congruent_equations}
    \begin{cases}
      a^2 d t^3 + ac t^2 + bt + 1 & \equiv 0 \pmod {p^3} \\
      act^2 + 2bt + 3 & \equiv 0 \pmod {p^2} \\
      3 + bt & \equiv 0 \pmod p 
    \end{cases}\,.
\end{equation}

\section{Розв'язання системи конгруентних рівнянь}
Розглянемо випадок, коли просте число $p > 3$. Маємо рівняння
\begin{align*}
bt \equiv p-3 \pmod p.
\end{align*}
Дане рівняння має розв'язок $\iff \gcd(b,p) \mid p-3$. Оскільки $p$ -- просте, то $\gcd(b,p) \in \{1,p\}$. Якби $\gcd(b,p) = p$, то отримали би $p \mid p-3$ -- неможливо.

Отже, якщо $p \nmid b$, то вищезгадане рівняння має розв'язок. При цьому оскільки $\gcd(b,p) = 1$, то рівняння має лише єдиний розв'язок вигляду
\begin{align*}
t \equiv -3b^{-1} \pmod p
\end{align*}

Єдиний можливий розв'язок необхідно підставити в інші два рівняння системи. Для початку піднімемо розв'язок до $\pmod {p^2}$ -- отримуємо наступне:
\begin{align*}
t \equiv -3b^{-1} + kp \pmod {p^2},
\end{align*}
де $k \in \{0,1,\dots,p-1\}$. Щойно отримане $t$ підставимо в друге рівняння:
\begin{align*}
act^2 + 2bt + 3 & \equiv ac(-3b^{-1} + kp)^2 + 2b(-3b^{-1}+kp) + 3 \\
& \equiv ac(k^2p^2 - 6kpb^{-1} + 9b^{-2}) + 2b(kp - 3b^{-1}) + 3 \\
& \equiv -6kp acb^{-1} + 9 acb^{-2} + 2bkp - 3 \equiv 0 \pmod {p^2}
\end{align*}
Домножимо обидві частини на $b^2$ -- отримуємо:
\begin{align*}
-6kpabc + 9ac + 2b^3kp - 3b^2 & \equiv 2bkp(b^2 - 3ac) - 3(b^2 - 3ac) \\
& \equiv (b^2 - 3ac)(2bkp -1) \equiv 0 \pmod {p^2}
\end{align*}
Зауважимо, що $\gcd(2bkp-1, p^2) = 1$, тому можемо скоротити рівняння:
\begin{align*}
b^2 - 3 ac \equiv 0 \pmod {p^2}
\end{align*}

Отже, якщо $p \nmid b$, а також $b^2 \equiv 3ac \pmod {p^2}$, то наш єдиний розв'язок задовольняє систему із останніх двох рівнянь. Причому $t \equiv -3b^{-1} + kp \pmod {p^2}$ не залежить від параметру $k$.

Зробимо аналогічно для першого рівняння системи, тільки замість $\pmod {p^3}$ розглядається $\pmod {p^2}$ -- маємо:
\begin{multline*}
a^2 dt^3 + act^2 + bt + 1 \\ \equiv a^2 d (-3b^{-1} + kp)^3 + ac (-3b^{-1} + kp)^2 + b (-3b^{-1} + kp) + 1 \\
\equiv a^2 d (-27 b^{-3} + 27kp b^{-2}) + ac (9b^{-2} - 6b^{-1}kp) + b(-3b^{-1} + kp) + 1 \\ \equiv 0 \pmod {p^2}
\end{multline*}
Домножимо обидві частини рівності на $p^{-3}$ -- із урахуванням рівності $b^2 \equiv 3 ac \pmod {p^2}$ отримуємо:
\begin{multline*}
a^2 d(-27 + 27 kpb) + ac(9b - 6b^2 kp) + b^4(-3b^{-1} + kp) + b^3 \\
\equiv -27a^2d  + 27 a^2 b d k p + 9abc - 6 ab^2 c kp - 3b^3 + b^4 kp + b^3 \\
\equiv -27a^2d + 27 a^2 b d k p + 3b^3 - 2 b^4 kp - 3b^3 + b^4 kp + b^3 \\
\equiv -27a^2d (1 - bkp) + b^3(1 - bkp) \equiv (1-bkp)(b^3 - 27a^2d) \equiv 0 \pmod {p^2}
\end{multline*}
Зауважимо, що $\gcd(1-bkp, p^2) = 1$, тому можемо скоротити рівняння:
\begin{align*}
b&3 - 27a^2 d \equiv 0 \pmod {p^2}
\end{align*}

Нарешті, розглянемо перше рівняння:
\begin{align*}
a^2 dt^3 + act^2 + bt + 1 \equiv 0 \pmod {p^3}
\end{align*}
Оскільки $t \equiv -3b^{-1} + kp \pmod {p^2}$ буде розв'язком системи останніх двох рівнянь при вищезгаданих умов, причому нема залежності від $k$, то для спрощення оберемо $t \equiv -3b^{-1} \pmod {p^2}$. Піднімемо це рівняння до $\pmod {p^3}$ -- отримуємо наступне:
\begin{align*}
t \equiv -3b^{-1} + lp^2 \pmod {p^2},
\end{align*} 
де $l \in \{0,1,\dots,p-1\}$. Щойно отримане $t$ підставимо в перше рівняння:
\begin{multline*}
a^2 dt^3 + act^2 + bt + 1 \\ 
\equiv a^2 d (-3b^{-1} + lp^2)^3 + ac(-3b^{-1} + lp^2)^2 + b(-3b^{-1} + lp^2) + 1 \\
\equiv a^2d (-27b^{-3} + 27b^{-2} lp^2) + ac (9b^{-2} - 6 l b^{-1} p^2) + b (-3b^{-1} + lp^2) + 1 \\ 
\equiv 0 \pmod {p^3}
\end{multline*}
Домножимо обидві частини рівності на $b^3$ -- отримуємо:
\begin{multline*}
a^2d (-27 + 27b lp^2) + ac (9b - 6 l b^2 p^2) + (-3 b^3 + lp^2 b^4) + b^3 \\ 
\equiv 0 \pmod {p^3}
\end{multline*}
Пригадаємо, що необхідно $b^2 \equiv ac \pmod {p^2}$, а також $27a^2 d \equiv b^3 \pmod {p^2}$. Підмінаючи до $\pmod {p^3}$, отримуємо:
\begin{align*}
3ac \equiv b^2 + \xi p^2 \pmod {p^3} \\
27a^2 d \equiv b^3 + \eta p^2 \pmod {p^3},
\end{align*}
де $\xi,\eta \in \{0,1,\dots,p-1\}$. Внаслідок чого
\begin{multline*}
27 a^2d (-1+ b lp^2) + 3ac (3b - 2 l b^2 p^2) + (-3 b^3 + lp^2 b^4) + b^3 \\ \equiv (b^3 + \eta p^2)(-1 + b lp^2) + (b^2 + \xi p^2)(3b - 2 l b^2 p^2) + (-3 b^3 + lp^2 b^4) + b^3 \\ \equiv -b^3 + b^4 lp^2 - \eta p^2 + 3b^3 - 2b^4 lp^2 + 3 \xi p^2 b - 3b^3 + b^4 lp^2 + b^3 \\
\equiv - \eta p^2 + 3\xi p^2 b \equiv 0 \pmod {p^3}
\end{multline*}
Отже, ми отримали наступне:
\begin{align*}
\eta p^2 \equiv 3 \xi p^2 b \pmod {p^3}
\end{align*}
Тоді зауважимо, що
\begin{align*}
27 a^2 d \equiv b^3 + \eta p^2 \equiv b^3 + 3 \xi p^2 b \equiv b^3 + 3b(3ac - b^2) \equiv -2b^3 + 9abc \pmod {p^3}
\end{align*}
Підсумуємо:

\begin{lemma}
\label{system_when_p_is_prime_and_bigger_than_three}
Нехай $p > 3$ -- просте число. Відомо, що $p \nmid b,\ b^2 \equiv 3ac \pmod {p^2}$, а також $27a^2d \equiv 9abc - 2b^3 \pmod {p^3}$. Тоді система конгруентних рівнянь \eqref{system_congruent_equations} має єдиний розв'язок $t \equiv -3 b^{-1} \pmod {p^3}$.
\end{lemma}

Пригадаємо, що кубічному кільцю відповідає кубічна форма, дискримінант якої обчислюється за формулою
\begin{equation}
\Delta = b^2 c^2 - 4c^3 a - 4 b^3 d + 18 abcd - 27a^2d^2
\end{equation}
Сприймаючи дискримінант за функцію від чотирьох змінних, знайдемо частинну похідну за аргументом $d$:
\begin{align*}
\Delta'_d = -4b^3 + 18 abc - 54a^2 d
\end{align*}
Тоді умову $27a^2d \equiv 9abc - 2b^3 \pmod {p^3}$ після множення на $2$ можна записати таким чином:
\begin{align*}
\Delta'_d \equiv 0 \pmod {p^3}
\end{align*}

\iffalse
Розпишемо матриці $[\omega_1],\ [\omega_2]$ оператора множення. Для множення на $\omega_1$ маємо
\begin{align*}
\omega_1 \cdot 1 = \omega_1 = 0 \cdot 1 + 1 \cdot \omega_1 + 0 \cdot \omega_2 \\
\omega_1 \cdot \omega_1 = -ac + b \omega_1 + a \omega_2 \\
\omega_1 \cdot \omega_2 = ad = ad \cdot 1 + 0 \cdot \omega_1 + 0 \cdot \omega_2
\end{align*}
Водночас для множення на $\omega_2$ маємо
\begin{align*}
\omega_2 \cdot 1 = \omega_2 = 0 \cdot 1 + 0 \cdot \omega_1 + 1 \cdot \omega_2 \\
\omega_2 \cdot \omega_1 = ad = ad \cdot 1 + 0 \cdot \omega_1 + 0 \cdot \omega_2 \\
\omega_2 \cdot \omega_2 = -bd + d \omega_1 + c 
\end{align*}
Звідси отримуємо матриці
\begin{align*}
[\omega_1] = \begin{pmatrix}
0 & -ac & ad \\
1 & b & 0 \\
0 & a & 0
\end{pmatrix},\quad [\omega_2] = \begin{pmatrix}
0 & ad & -bd \\
0 & 0 & d \\
1 & 0 & c
\end{pmatrix}
\end{align*}
\fi

\newpage

\chapter{Назва розділу}\label{chaper2}

\section{Назва секції}\label{section2.1}


\newpage

\phantomsection
\chapter*{Висновки}
\addcontentsline{toc}{chapter}{Висновки}

У роботі ??????

\newpage

\phantomsection
\renewcommand{\bibname}{Список використаних джерел}

\begin{thebibliography}{99}
\addcontentsline{toc}{chapter}{Список використаних джерел}
\itemsep=0pt

\bibitem{boyko-thesis}
Бойко В.М.,
Узагальненi оператори Казiмiра,
сингулярнi модулi редукцiї
та симетрiї диференцiальних рiвнянь,
Дис. \dots\ док. фіз.-мат. наук,  Інституту математики НАН України, Київ, 2018, 338~с., \url{https://www.imath.kiev.ua/~boyko/BoykoThesis.pdf}.



\bibitem{boyko2021}
Boyko V.M., Lokaziuk O.V., Popovych R.O.,
Admissible transformations and Lie symmetries of linear systems of second-order ordinary differential equations, \href{https://arxiv.org/abs/2105.05139}{arXiv:2105.05139}.

\bibitem{Maple17}
Maple 17, \url{https://www.maplesoft.com/products/Maple/}.


\bibitem{Olver1995}
Olver P.J., Equivalence, invariants, and symmetry, Cambridge, University Press Cambridge, 1995, xvi+525~pp.,
\url{https://doi.org/10.1017/CBO9780511609565}.


\bibitem{PopovychBoykoNesterenkoLutfullin2003}
Popovych R.O., Boyko V.M., Nesterenko M.O., Lutfullin M.V., Realizations of real low-dimensional Lie algebras, \textit{J.~Phys.~A} \textbf{36} (2003), no.~26,
7337--7360, \url{https://doi.org/10.1088/0305-4470/36/26/309}; \href{https://arxiv.org/abs/math-ph/0301029}{math-ph/0301029}.



\end{thebibliography}

\appendix

\chapter{Назва додатку}\label{appendix1}

\section{Назва секції додатку}\label{appendix1.1}

\end{document}

\newpage

\bibliographystyle{plain} %plain %sigma %amsalpha %ugost2008
\bibliography{ref}



